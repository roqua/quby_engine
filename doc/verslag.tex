\documentclass[10pt,fleqn,a4paper]{article}
\usepackage[english]{babel}
\usepackage{graphicx} % plaatjes zijn goed
\usepackage{comment}  % uitcommentarieren van blokken oud spul
\usepackage[hmargin=2.5cm,vmargin=2cm]{geometry} % meer ruimte
\usepackage{amsmath,amssymb,latexsym}
\usepackage{float}
\usepackage{listings}
\lstset{language=Ruby}

\newcommand{\keywords}[1]{\par\addvspace\baselineskip\noindent\enspace\ignorespaces{\textbf{Keywords:~}}#1}
\newcommand{\negmuchspace}{\negthinspace\negthinspace\negthinspace\negthinspace\negthinspace\negthinspace\negthinspace\negthinspace\negthinspace\negthinspace\negthinspace\negthinspace\negthinspace\negthinspace\negthinspace\negthinspace\negthinspace\negthinspace}
\newcommand{\ds}[1]{d^2_{#1}}
\newcommand{\dsa}[1]{\overline{d^2}_{#1}}

%opening
\title{Domain-Specific Questionnaires}
\author{Marten Veldthuis}
\addtolength{\parskip}{0.5\baselineskip}
\allowdisplaybreaks[1]

\begin{document}

\sloppy
\begin{twocolumn}
\twocolumn[
\maketitle
\begin{@twocolumnfalse}
\begin{abstract}
This paper has an abstract. But I will not write the abstract until
the paper is largely done. Makes sense, no?  

\keywords{Domain Specific Languages, Language Design, Ruby, Rails,
  Dynamic Programming, Questionnaires, Psychiatrics, Web-Applications}

\end{abstract}
\vspace{5mm}
\end{@twocolumnfalse}
]

\section{Introduction}
% 1 - context of problem

The Rob Giel Onderzoekscentrum (RGOc) is a research center developing
a product called RoQua. RoQua is a web application which allows
psychiatric departments of health-care providers to monitor patients by
periodically letting them fill out questionnaires over the Internet.

Right now, RoQua uses a third-party web service called GlobalPark to
display, and have patients fill in, the actual questionnaires. While
this is a good separation of concerns for the application, in section
\ref{sec:problems} we will outline the problems with GlobalPark. In
section \ref{sec:approach} we explain our solutions to these problems,
and the rationale behind them.

\section{Problems}\label{sec:problems}
%   - which problem is addressed

GlobalPark is a costly service, and given the nature of the data, the
RGOc would rather not be reliant on a service which is not under its
own control.

Another problem the RGOc is experiencing with the GlobalPark service
is a matter of usability. Defining questionnaires through the web
interface of GlobalPark is a slow process. The web interface for
editing questionnaires is slow, and gets increasingly slower when
there are more questions defined for a questionnaire. 

It's also not possible to define multiple views for the same
questionnaire. In RoQua, this is needed because for most
questionnaires, both a patient-version as well as a bulk-input version
needs to be available. Currently, this means questionnaires have to be
defined twice, both in GlobalPark and in RoQua itself.

Lastly, the way GlobalPark sends back information to RoQua is by doing
an HTTP GET request back to RoQua. This opens up a potential CSRF
security hole. Additionally, this interface contains character
encoding bugs, and GlobalPark is largely unresponsive to bug reports.

All things considered, the RGOc would like to move away from
GlobalPark and switch to a replacement of their own.

\section{Related work}
% 2 - related work
%     - for DSL
%     - for webbased forms: globalpark, netquestionnaire, but also
%       indirect competitors


\section{Goals}

The goals for the project, which was given the name Quby, and will be
referred to as such from here on, are then defined as:

\begin{enumerate}
\item Replace all functionality of GlobalPark, for as far as we use it
  currently.
\item \label{dsl} Make it easier to define questionnaires.
\item \label{restful} Make the API interface with RoQua simpler.
\end{enumerate}


\section{Background of technology}
%   - background of technology used
Roqua is written using a web application framework called Ruby on Rails. Given that this framework provides options for integrating with other Ruby on Rails applications, Quby will be written in Rails too. Communication between the two is done by exposing a RESTful web service interface on the side of Quby, which will output JSON documents.


\section{Architectural Overview}
% 3 - architectural overview (of Quby, and of Quby as a service in its setting within Roqua)

Ruby on Rails applications follow a standard Model-View-Controller pattern.

\subsection{DSL}

One issue traditionnally seen in many Ruby domain-specific languages is that they tend to pollute the namespace. For instance, take the following snippet of Rails code:

\begin{lstlisting}
class Sprocket < ActiveRecord::Base
  belongs_to :widget
end
\end{lstlisting}

In this case, \lstinline{belongs_to} is part of the DSL which Rails' ActiveRecord provides. But it's actually a class method, and if you try to define your own \lstinline{belongs_to}, you have to be careful. For Rails, this is usually not that big a problem, given the naming of the DSL methods they use.

However, with the DSL we wanted to have for Quby, these collisions would be common. In this paper we will show a method of solving this problem, without having to consort to Ruby parsetree manipulation tools, which can be cumbersome to build, maintain and debug.

Our approach is similar in spirit to the Decorator pattern\cite{Erich:1995design} in that we too have a seperate Questionnaire\-Decorator class which decorates the Questionnaire instances at runtime. Implementation-wise, we leverage Ruby's dynamic nature.

Upon instantiation of a questionnaire object we instantiate a decorator and evaluate the questionnaire definition within the instance. Any DSL methods are implemented as instance methods on the decorator. These DSL methods then modify the questionnaire object, filling in instance variables such as \lstinline{@questions}.

\section{Implementation}
% 4 - Implementation



\section{Evaluation}
% 5 - evaluation
%     - user evaluation
%     - comparison with tools


\section{Future Work}


\section{Conclusion}
% 6 concluding remarks


%----------------------------------------------------------------------------------
\bibliographystyle{plain}
\bibliography{DomainSpecificQuestionnaires}

\end{twocolumn}

\end{document}
