%% LyX 1.6.7 created this file.  For more info, see http://www.lyx.org/.
%% Do not edit unless you really know what you are doing.
\documentclass[oneside,english]{book}
\usepackage[T1]{fontenc}
\usepackage[latin9]{inputenc}
\usepackage{listings}
\setcounter{secnumdepth}{3}
\setcounter{tocdepth}{3}
\usepackage{babel}

\begin{document}

\title{Quby gebruikershandleiding}


\author{Marten Veldthuis}


\date{\today}

\maketitle

\chapter{Inleiding tot Quby}

Quby is een systeem om vragenlijsten in te ontwerpen.

In Quby worden vragenlijsten gedefinieerd met behulp van een simpele
tekst, kwa idee vergelijkbaar met bijvoorbeeld een SPSS script. Hierbij
zijn de volgende algemene dingen van belang:
\begin{description}
\item [{Commentaar}] Regels die met een hekje (\#) beginnen worden gezien
als commentaar, en worden door Quby genegeerd.
\item [{Strings}] Teksten moeten tussen dubbele aanhalingstekens worden
geplaatst. Als u binnen een tekst dubbele aanhalingstekens wil gebruiken,
dan moet u deze \emph{escapen} door er een backslash (\textbackslash{})
voor te zetten.
\end{description}

\chapter{Opbouw van een vragenlijst}

Een vragenlijst geeft allereerst zijn eigen naam aan. Dit gebeurt
via het \texttt{title} commando.


\begin{lstlisting}
title "Demonstratievragenlijst"
\end{lstlisting}


Hierna volgen de definities van de pagina's. Een pagina wordt in Quby
aangeduid met het \texttt{panel} commando.


\begin{lstlisting}
panel do
  # Hier komen de definities van de vragen voor dit panel
end
\end{lstlisting}



\end{document}
